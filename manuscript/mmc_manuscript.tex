\documentclass[11pt, letterpaper]{article}   	% use "amsart" instead of "article" for AMSLaTeX format
%\usepackage{geometry}                		% See geometry.pdf to learn the layout options. There are lots.
%\geometry{letterpaper}                   		% ... or a4paper or a5paper or ... 
%\geometry{landscape}                		% Activate for rotated page geometry
%\usepackage[parfill]{parskip}    		% Activate to begin paragraphs with an empty line rather than an indent
\usepackage{graphicx}				% Use pdf, png, jpg, or eps§ with pdflatex; use eps in DVI mode
								% TeX will automatically convert eps --> pdf in pdflatex		
\usepackage{amssymb}
\usepackage{amsmath}

%SetFonts

\newcommand{\eq}[1]{Eq.~(\ref{#1})}
\newcommand{\Eq}[1]{Equation~(\ref{#1})}
\newcommand{\fig}[1]{Fig.~\ref{#1}}
\newcommand{\Fig}[1]{Figure~\ref{#1}}

\title{Distinguishing among coalescent models using two-site joint allele frequency spectra}
\author{Daniel P. Rice}
\date{\today}							

\begin{document}
\maketitle

\abstract{The genetic diversity of a population reflects its demographic and
evolutionary history. Methods for inferring this history typically
assume that the ancestry of a sample can be modeled by the Kingman
coalescent process. A defining feature of the Kingman coalescent is
that it generates genealogies that are binary trees: no more than two
ancestral lineages may coalesce at the same time. However, this
assumption breaks down under several scenarios. For example, pervasive
natural selection, rapid spatial range expansion, and extreme
variation in offspring number can all generate genealogies with
"multiple-merger" events in which more than two lineages coalesce
instantaneously. Therefore, detecting multiple mergers is important
both for understanding which forces have shaped the diversity of a
population and for avoiding fitting misspecified models to data.
Current methods to detect multiple mergers rely on the average site
frequency spectrum (SFS). However, the signatures of multiple
mergers in the average SFS are also consistent with a Kingman
coalescent process with a time-varying population size. Here, I
present a new method for detecting multiple mergers based on
the mutual information of the joint site frequency spectrum at pairs of linked sites. Unlike the average SFS,
the mutual information depends mostly on the topologies of genealogies
rather than their branch lengths and is therefore robust to most
demographic effects.}

\section*{Introduction}
A typical goal of population genetics is to infer the evolutionary and demographic history of a population from its contemporary genetic diversity. The genetic differences among individuals reflect their genealogical history, which, in turn, reflects the history of various population parameters such as the population size, spatial structure, and the influence of natural selection. By modeling how the (unobserved) genealogical process depends on these parameters and how patterns of genetic diversity depend on genealogy, we can develop inference procedures to learn about the past.

The most commonly-used model is the Kingman coalescent, which arises in many models of neutral evolution.
The key characteristic of the Kingman coalescent is that only two lineages are permitted to share a common ancestor at the same point in (continuous) time.
As a result, the genealogies generated by the Kingman coalescent are randomly bifurcating trees: each node subtends exactly two branches and the number of leaves subtended by one of these branches is uniformly distributed.

Kingman coalescent models are appropriate when a single timescale controls patterns of relatedness between individuals in a sample from a population .
This timescale, which is inversely proportional to the rate of coalescence between pairs of lineages, determines the branch lengths of genealogies and thus determines observable quantities such as the average genetic diversity between pairs of individuals and length distribution of genomic segments that are identical-by-descent.
Population genetic inference methods interpret these quantities in light of the Kingman distribution of tree topologies to estimate the coalescent timescale.
In the simplest models of neutral evolution, the coalescent timescale is proportional to the number of individuals in the population. Therefore, it is commonly referred to as the \emph{effective population size}


%The Kingman coalescent generates genealogies with several distinguishing characteristics. First, the genealogies are binary trees: each node has only two children. A second and related topological feature is that for a node subtending $n$ leaves, the number of leaves subtended by one of its children is uniformly distributed between 1 and $n-1$. Finally, under a neutral model of evolution, the rate of coalescence (and thus the branch lengths of the genealogy) is inversely proportional to the population size.

The site frequency spectrum (SFS) is a useful summary statistic of genetic data. Defined as the fraction of mutations as a function of allele frequency, the SFS contains a lot of information about the distribution of underlying genealogies. The number of mutations in $i$ samples is proportional to the lengths of branches subtending $i$ leaves of the genealogy. This quantity depends on the topology and branch lengths of trees. Assuming the Kingman model is true, one can infer the history of the population size from the SFS because the number of mutations at a particular frequency is related to the coalescent rate as a function of time. For example, high-frequency alleles are the result of mutations on deep branches and thus reflect the population size in the distant past. 

A serious problem for this inference procedure is that alternative models of evolution generate different coalescent models. For example, models of pervasive weak selection, skewed offspring number distributions, and periodic strong bottlenecks can all generate genealogies that differ from the Kingman both topologically and in terms of branch length.
Collectively, these models are known as \emph{multiple mergers coalescents} because, unlike the Kingman coalescent, they permit multiple lineages to coalesce instantaneously. As a result, multiple-mergers genealogies contain nodes with more than two children and lack the uniform branching structure of the Kingman. Even more important for inference is the fact that the coalescent timescales---and thus the branch lengths and levels of diversity---are determined by parameters other than the population size.
Thus, the results of a Kingman-based inference procedure applied to data generated by a non-Kingman process will be both quantitatively and qualitatively wrong.
Clearly, it would be useful to be able to distinguish among different coalescent models, both to check that popular inferences of historical population sizes are valid and to understand which evolutionary forces are dominant in various populations.

An obvious question is whether we can use the average site frequency spectrum to reject the Kingman? There are two distinguishing features of the SFS generated by multiple mergers coalescents. First, low-frequency mutations are enriched relative to the time-homogeneous Kingman coalescent. Unfortunately, this is also a feature of the Kingman coalescent in growing populations, so it is not useful for model selection. Second, the multiple-merger SFS is non-monotonic: it has a positive slope for high frequencies. Unlike the excess of rare alleles, this feature cannot be reproduced by an exchangeable Kingman coalescent model. However, it is difficult to identify the non-monotonicity in real data. The problem is that it necessary to know which allele is ancestral to distinguish frequency $x$ mutations from frequency $1-x$ mutations. Various methods for identifying the ancestral allele exist, but in practice even a very low rate of error can generate a spurious signal.

Ideally, we would like a statistic (or statistics) that:
\begin{enumerate}
\item is qualitatively different between models of genealogies,
\item is robust to demographic changes, and 
\item requires minimal information beyond genetic diversity.
\end{enumerate}
The first two criteria are important for making qualitative distinctions between models. The last one would maximize its applicability.
Here, I propose that the correlations between elements of the site frequency spectrum are such a statistic.

\end{document}

